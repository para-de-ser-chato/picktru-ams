\documentclass[12pt,a4paper]{article}
\usepackage[a4paper, portrait, margin=1in]{geometry}
\usepackage{hyperref}
\usepackage{fancyhdr}
\usepackage{graphicx}
\usepackage{pdflscape}
\usepackage{svg}
\usepackage{wrapfig}
\usepackage{lastpage}
\usepackage[autostyle]{csquotes}

\MakeOuterQuote{"}
\graphicspath{ {./} }
\hypersetup{
	colorlinks=true,
	linkcolor=blue,
	filecolor=magenta,
	urlcolor=blue,
	pdftitle={Projeto AMS 2022/2023 - Entrega 2 - Grupo 38},
	pdfpagemode=FullScreen,
}

\pagestyle{fancy}
\fancyhf{}
\fancyhead[C]{Projeto AMS 2022/2023 - Entrega 2 - Grupo 38}
\fancyfoot[C]{99209, 99256, 99311}
\fancyfoot[R]{Pág. \thepage{} de \pageref*{LastPage}}

\begin{document}

\begin{titlepage}
	\begin{center}

		\vspace*{0.5cm}
		\includegraphics[width=0.5\textwidth]{ist-logo.png}

		\vspace{1cm}
		\Huge
		\textbf{Projeto de AMS: Entrega 2}

		\vspace{0.5cm}
		\LARGE
		Grupo 38 \quad | \quad Turno L08 \quad | \quad 2022/2023

		\vfill
	\end{center}

	\large
	\begin{tabular}{c|c|c}
		\textbf{Nome}           & \textbf{Número} & \textbf{Est. Horas Trabalho} \\
		\hline
		Diogo Romão Cardoso     & 99209           & 20h                          \\
		João Fernandes Rocha    & 99256           & 20h                          \\
		Rafael Serra e Oliveira & 99311           & 20h
	\end{tabular}

	\vspace{5cm}
\end{titlepage}

\begin{landscape}
	\section*{B.1.1. Diagrama de Vista Geral do Negócio}
	\includegraphics[width=1.59\textwidth]{../ArchiMate.png}
\end{landscape}

\begin{landscape}
	\section*{B.1.2. Diagrama do Processo P-SET}
	\includegraphics[width=1.59\textwidth]{../BPMN_P-SET.png}
\end{landscape}

\begin{landscape}
	\section*{B.1.2. Diagrama do Processo P-ON}
	\includegraphics[width=1.59\textwidth]{../BPMN_P-ON.png}
\end{landscape}

\begin{landscape}
	\section*{B.2.1. Diagrama de Casos de Uso: Sistema RUN}
	\includegraphics[width=1.59\textwidth]{../UC_RUN.png}
\end{landscape}

\begin{landscape}
	\section*{B.2.2. Diagrama de Casos de Uso: Sistema STORE}
	\includegraphics[width=1.19\textwidth]{../UC_STORE.png}
\end{landscape}

\begin{landscape}
	\section*{B.3. Diagrama de Classes: Aplicação RUN}
	\includegraphics[width=1.46\textwidth]{../Class_RUN.png}
\end{landscape}

\begin{landscape}
	\section*{B.4. Diagrama de Máquina de Estados: Zona (Aplicação RUN)}
	\includegraphics[width=1.59\textwidth]{../State_Zona.png}
\end{landscape}

\begin{landscape}
	\section*{B.5. Diagrama de Blocos: Sistema STORE}
	\includegraphics[width=1.59\textwidth]{../BDD_STORE.png}
\end{landscape}

\begin{landscape}
	\section*{B.6. Diagrama Interno de Blocos: Sistema STORE}
	\includegraphics[width=1.31\textwidth]{../iBD_STORE.png}
\end{landscape}

\section*{Notas, \emph{Assumptions} e Escolhas de Desenho} \label{notes}
\normalsize
\textbf{B.1.1. Diagrama de Vista Geral do Negócio}
\begin{itemize}
	\item Modelou-se o P-ON como um Processo de Negócios que se serve de uma Interação de Negócios (que lhe fornece a sua funcionalidade), visto a designação explícita de "Processo" imposta pelo Universo de Discurso, apesar de ser possível modelar o P-ON como uma Interação de Negócios por si só (sem a necessidade de um Processo de Negócios).
	\item Optou-se por representar uma relação "servir" entre a Aplicação TEST e os Interfaces Aplicacionais para interação com os sistemas do cliente, visto que esta "fornece a sua funcionalidade" a esses elementos (durante a execução do processo P-ON).
\end{itemize}
\textbf{B.1.2. Diagrama do Processo P-SET}
\begin{itemize}
	\item Considerou-se que os detalhes das tarefas a executar durante a fase de instalação estão definidos no Plano de Projeto, tendo validado com a docente a notação abreviada de acesso relativa ao sub-processo inteiro, de modo a evitar a complexidade desnecessária do diagrama.
	\item Após deliberação, considerou-se que os relatórios referidos nas linhas 86-88 e nas linhas 99-102 do Universo de Discurso são os mesmos (Relatório Final), dado a semelhança da linguagem utilizada e do procedimento expresso, levando, caso se considerassem documentos distintos, a uma repetição desnecessária. Isto é também reforçado pela atomicidade da fase de testes aludido pelas linhas 80-11 do Universo de Discurso.
\end{itemize}
\textbf{B.1.3. Diagrama do Processo P-ON}
\begin{itemize}
	\item Seguindo o aconselhado pela docente, optou-se por modelar o processo com um \emph{Message Start Event} em cada \emph{pool}, visto que o Universo de Discurso nada diz relativamente à origem do Plano de Trabalhos, nem se as unidades referidas o recebem em simultâneo.
	\item Conforme o exemplificado nas aulas teóricas, utilizaram-se \emph{Parallel Gateways} em pontos críticos de comunicação, prevenindo problemas de (falta de) sincronização entre os fluxos dos dois participantes, de modo a assegurar que nenhuma mensagem é perdida.
	\item Não sendo o Universo de Discurso explícito, e após validar com a docente, assumiu-se que não seria necessário a \emph{U-HW} aguardar que a \emph{U-SW} terminasse de analisar o plano de trabalhos antes da primeira se deslocar à loja, nem que seria necessário que a \emph{U-SW} aguardasse que a \emph{U-HW} terminasse de executar as suas tarefas na loja antes da primeira começar a preparação dos testes.
\end{itemize}

\end{document}